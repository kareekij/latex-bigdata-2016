
\section{Related Works}
Because network data is growing rapidly, there is a large literature on network sampling.  Sampling is necessary for two reasons: (1) Current computing power cannot always handle the sizes of available data, and (2) Collecting data is time-consuming.  For example, it is impossible to collect the whole Facebook network within a reasonable amount of time. %These are the reasons why network sampling is important and gain a lot of attention.

Accordingly, network sampling can be separated into two main scenarios.  The first scenario is referred to as ``scaling-down" or ``down-sampling". In this scenario, the entire network data is available, and the goal is to scale down the network to some desired size, and the sample graph should preserve the network properties as a representative of the original network.  The second scenario, and the focus of this paper, is a data collection scenario where we have a limited view of the network.  Here, decisions about how to grow the sample are based only on the data that has been observed so far.

In \cite{leskovec2006sampling}, the authors study the characteristics of different sampling algorithms. They evaluate how well the output graphs from different algorithms capture network properties. Similarly, Maiya and Berger-Wolf present an algorithm that aims to preseve the community structure of the original network~\cite{maiya2010sampling}. This algorithm is built on the concept of expander graphs. The results shows that sample is capable of capturing the community structure.

The most relevant work to this paper is presented in~\cite{avrachenkov2014pay}.  This work introduces a greedy sampling approach called Maximum Observed Degree (MOD). The goal of this algorithm is to maximize the network coverage, which is as same as our objective. In MOD, in each step the algorithm selects the node with the largest observed degree in the sample. Their experimental results show that MOD outperforms other algorithms such as BFS, DFS and RW. To the best of our knowledge, MOD is currently the best algorithm in this class.

% An example of a floating figure using the graphicx package.
% Note that \label must occur AFTER (or within) \caption.
% For figures, \caption should occur after the \includegraphics.
% Note that IEEEtran v1.7 and later has special internal code that
% is designed to preserve the operation of \label within \caption
% even when the captionsoff option is in effect. However, because
% of issues like this, it may be the safest practice to put all your
% \label just after \caption rather than within \caption{}.
%
% Reminder: the "draftcls" or "draftclsnofoot", not "draft", class
% option should be used if it is desired that the figures are to be
% displayed while in draft mode.
%
%\begin{figure}[!t]
%\centering
%\includegraphics[width=2.5in]{myfigure}
% where an .eps filename suffix will be assumed under latex, 
% and a .pdf suffix will be assumed for pdflatex; or what has been declared
% via \DeclareGraphicsExtensions.
%\caption{Simulation Results}
%\label{fig_sim}
%\end{figure}

% Note that IEEE typically puts floats only at the top, even when this
% results in a large percentage of a column being occupied by floats.


% An example of a double column floating figure using two subfigures.
% (The subfig.sty package must be loaded for this to work.)
% The subfigure \label commands are set within each subfloat command, the
% \label for the overall figure must come after \caption.
% \hfil must be used as a separator to get equal spacing.
% The subfigure.sty package works much the same way, except \subfigure is
% used instead of \subfloat.
%
%\begin{figure*}[!t]
%\centerline{\subfloat[Case I]\includegraphics[width=2.5in]{subfigcase1}%
%\label{fig_first_case}}
%\hfil
%\subfloat[Case II]{\includegraphics[width=2.5in]{subfigcase2}%
%\label{fig_second_case}}}
%\caption{Simulation results}
%\label{fig_sim}
%\end{figure*}
%
% Note that often IEEE papers with subfigures do not employ subfigure
% captions (using the optional argument to \subfloat), but instead will
% reference/describe all of them (a), (b), etc., within the main caption.


% An example of a floating table. Note that, for IEEE style tables, the 
% \caption command should come BEFORE the table. Table text will default to
% \footnotesize as IEEE normally uses this smaller font for tables.
% The \label must come after \caption as always.
%
%\begin{table}[!t]
%% increase table row spacing, adjust to taste
%\renewcommand{\arraystretch}{1.3}
% if using array.sty, it might be a good idea to tweak the value of
% \extrarowheight as needed to properly center the text within the cells
%\caption{An Example of a Table}
%\label{table_example}
%\centering
%% Some packages, such as MDW tools, offer better commands for making tables
%% than the plain LaTeX2e tabular which is used here.
%\begin{tabular}{|c||c|}
%\hline
%One & Two\\
%\hline
%Three & Four\\
%\hline
%\end{tabular}
%\end{table}


% Note that IEEE does not put floats in the very first column - or typically
% anywhere on the first page for that matter. Also, in-text middle ("here")
% positioning is not used. Most IEEE journals/conferences use top floats
% exclusively. Note that, LaTeX2e, unlike IEEE journals/conferences, places
% footnotes above bottom floats. This can be corrected via the \fnbelowfloat
% command of the stfloats package.

